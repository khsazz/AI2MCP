% ============================================================================
% Chapter 3: Methodology
% ============================================================================

\chapter{Methodology}
\label{ch:methodology}

This chapter delineates the architectural design of the MCP-to-ROS~2 bridge and the Kinematic-GNN reasoning engine. By synthesizing standardized communication protocols with graph-based relational learning, this framework provides a substrate for decoupling high-level robotic intelligence from physical actuation while ensuring safety through pre-execution simulation \cite{roth_learned_2025, nguyen_model-free_2024, fisac_general_2017, liang_point_2025}.

\section{System Architecture and Theoretical Design}
\label{sec:system-architecture}

The architecture implements a multi-layered data flow designed to resolve the $N \times M$ integration problem \cite{zeng_m3llm_2025}. As illustrated in \Cref{fig:architecture}, the system is partitioned into four primary functional layers, ensuring that modifications to the AI ``brain'' or the robotic ``body'' do not require a full-stack reconfiguration \cite{liskova_cps-based_2025, imran_safety-critical_2025, zhou_multi-robot_2022}:

\begin{figure}[htbp]
    \centering
    \includegraphics[width=0.85\textwidth]{figures/architecture.png}
    \caption{MCP-ROS~2 Bridge architecture showing the flow from AI agents through the MCP protocol to the ROS~2 control stack. The GNN reasoner serves as the central semantic engine.}
    \label{fig:architecture}
\end{figure}

\begin{enumerate}
    \item \textbf{Cognitive Reasoning Layer}: Houses cloud-hosted or local Large Language Models (e.g., Qwen 2.5, Llama 3.2) that generate action plans and process semantic graph feedback \cite{fu_rosbag_2025, taghian_explainability_2024, zhou_multi-robot_2022}.
    \item \textbf{Protocol Abstraction Layer}: An MCP stack utilizing Server-Sent Events (SSE) and JSON-RPC~2.0 transport to expose robot affordances through a standardized interface \cite{fu_rosbag_2025, liskova_cps-based_2025}. 
    \item \textbf{Robotic Integration Layer (Bridge)}: A stateful intermediary that translates MCP tool calls into ROS~2 topics, services, and actions.
    \item \textbf{Geometric Reasoning Layer}: A Kinematic-GNN reasoning engine that transforms raw 14-DoF kinematic streams into structured spatial predicates.
\end{enumerate}

\subsection{Design Principles for Decoupled Control}

Our architecture adheres to three primary design principles to operationalize swappable intelligence:
\begin{description}
    \item[Semantic Decoupling] The AI agent interacts exclusively with abstract \textit{Tools} and \textit{Resources}, possessing no intrinsic knowledge of the ROS~2 computational graph or DDS-level messaging.
    \item[Operational Transparency] Every tool call is logged with structured arguments and results, providing a protocol-level audit trail for safety monitoring and debugging \cite{fisac_general_2017, nguyen_model-free_2024, roth_learned_2025, liang_point_2025}.
    \item[Hardware Agnosticism] Robotic capabilities are modularized as tool handlers. This allows the system to swap hardware by simply updating the tool registration at the bridge \cite{liskova_cps-based_2025, fu_rosbag_2025, ding_stacked_2025}.
\end{description}

\section{MCP Tool Design}
\label{sec:mcp-tool-design}

Robot capabilities are exposed to the AI model as 13 standardized MCP tools, categorizing low-level control and sensing into semantically meaningful families \cite{fu_rosbag_2025, liskova_cps-based_2025, zhou_multi-robot_2022}.

\subsection{Motion and Perception Tool Families}
Motion tools encapsulate actuation primitives as asynchronous MCP actions. Perception tools query sensor states and aggregate high-frequency data into compact summaries.

\begin{table}[htbp]
\centering
\caption{Core Robotic Tool Families exposed via MCP}
\label{tab:tools}
\begin{tabular}{llp{6.5cm}}
\toprule
\textbf{Category} & \textbf{Tool Name} & \textbf{Description} \\
\midrule
\multirow{2}{*}{Motion} & \texttt{move\_forward} & Move forward by specified distance at a set speed \\
                        & \texttt{rotate} & Rotate the robot by a specified angle in degrees \\
\midrule
\multirow{2}{*}{Perception} & \texttt{get\_obstacle\_distances} & Returns distance to obstacles segmented by direction \\
                            & \texttt{scan\_surroundings} & Provides a 360° summary of detected obstacles \\
\bottomrule
\end{tabular}
\end{table}

\subsection{Prediction and Reasoning Tools}
These tools leverage the GNN for semantic reasoning and pre-execution simulation \cite{taghian_explainability_2024, zuo_graph-based_2021, zhou_multi-robot_2022, roth_learned_2025, jiang_learning_2024}. The \texttt{simulate\_action} tool provides an \texttt{EXECUTE} or \texttt{REPLAN} recommendation based on physical feasibility.

\section{MCP Resource Design}
\label{sec:mcp-resource-design}

Resources provide read-only access to robot and environment states via semantic URIs, ensuring the agent maintains a belief state without polling \cite{fu_rosbag_2025, zhou_multi-robot_2022, taghian_explainability_2024}.

\begin{table}[htbp]
\centering
\caption{MCP Resources for state monitoring}
\label{tab:resources}
\begin{tabular}{lp{8cm}}
\toprule
\textbf{URI} & \textbf{Content} \\
\midrule
\texttt{robot://pose} & Current Cartesian position ($x, y, z$) and orientation ($\theta$) \\
\texttt{robot://status} & Connectivity and movement status flags \\
\texttt{robot://world\_graph} & Full Kinematic-GNN scene graph (JSON) \\
\texttt{robot://lerobot/predicates} & Current list of active semantic predicates \\
\bottomrule
\end{tabular}
\end{table}

\section{Kinematic-GNN Architecture}
\label{sec:relational-gnn-arch}

The Kinematic-GNN processes kinematic states into relational predicates, enabling conceptual planning rather than raw coordinate manipulation \cite{taghian_explainability_2024, zuo_graph-based_2021, chen_relgnn_2025, zhou_multi-robot_2022, jiang_learning_2024}.

\subsection{Graph Construction}
Given a robot state vector $\mathbf{s} \in \mathbb{R}^{14}$ (representing the ALOHA system), we construct a graph $G = (V, E)$. Node features $\mathbf{x}_i$ are derived from forward kinematics:
\begin{equation}
    \mathbf{x}_i = [x_i, y_i, z_i, \theta_i, t_i]
\end{equation}
where $(x, y, z)$ is the 3D position, $\theta$ is the joint angle, and $t$ is a one-hot joint type encoding.

\subsection{Message Passing and Global Feature Conditioning (GFC)}
The architecture employs three GATv2 layers with residual connections. To accurately predict interaction predicates (e.g., \texttt{is\_holding}), we utilize \textbf{Global Feature Conditioning}, concatenating the global robot state $\mathbf{u}$ (e.g., gripper aperture) into the predicate head:
\begin{equation}
    P(\text{pred}_k | e_{ij}, \mathbf{u}) = \sigma(\text{MLP}_\text{pred}([\mathbf{h}_i \| \mathbf{h}_j \| \mathbf{u}]))
\end{equation}
This grounding ensures relational predictions are physically constrained by the robot's end-effectors.

\section{Predictive Temporal Reasoning (ST-GNN)}
\label{sec:stgnn}

To mitigate predicate flickering caused by sensor noise, we implement a \textbf{Spatiotemporal GNN (ST-GNN)}. This architecture incorporates a Gated Recurrent Unit (GRU) layer following the base GNN encoder to maintain temporal memory over a sequence of 5 frames.

\begin{figure}[htbp]
    \centering
    \includegraphics[width=0.85\textwidth]{figures/stgnn_architecture.png}
    \caption{Spatiotemporal GNN architecture: Integration of GRU layers with the Kinematic-GNN to ensure proactive temporal stability.}
    \label{fig:stgnn}
\end{figure}

\section{Methodological Comparison: Kinematic-GNN vs. Multimodal-GNN}
\label{sec:architecture-comparison}

We compare the baseline kinematic-geometric reasoning architecture (\textbf{Kinematic-GNN}) with a vision-augmented architecture (\textbf{Multimodal-GNN}) which utilizes visual features via a DINOv2 backbone.

\begin{description}
    \item[Kinematic-GNN] Relies exclusively on 14-DoF joint states and geometric projection. This lightweight approach assumes objects can be localized via standard detection or known poses.
    \item[Multimodal-GNN] Fuses visual embeddings from a pre-trained Vision Transformer (DINOv2) with kinematic node features via cross-attention. This aims to capture visual context not present in the joint state.
\end{description}

As shown in \Cref{fig:design_latency}, the Kinematic-GNN's $16\times$ speed advantage justifies its selection for real-time control.

\begin{figure}[htbp]
    \centering
    \includegraphics[width=0.85\textwidth]{figures/inference_latency.png}
    \caption{Methodological Latency Analysis: Kinematic-GNN achieves 1.5ms inference vs. 24ms for Multimodal-GNN, enabling control loops at 600+ Hz.}
    \label{fig:design_latency}
\end{figure}

\section{Safety Verification via Forward Dynamics}
\label{sec:forward-dynamics}

To address LLM non-determinism, a \textbf{ForwardDynamicsModel} performs rollouts to verify plans \cite{roth_learned_2025, nguyen_model-free_2024, fisac_general_2017, molnar_model-free_2021, liang_point_2025, imran_safety-critical_2025}.
\begin{equation}
    [G_{t+1}, c] = \text{DynamicsNet}(G_t, a_t)
\end{equation}
The bridge uses the predicted world graph $G_{t+1}$ and confidence $c$ to reject unsafe commands.

\section{Predicate Definitions}
\label{sec:predicate-definitions}

The Kinematic-GNN predicts nine binary predicates organized into two categories. These form the basis for the agent's environmental understanding.

\begin{table}[htbp]
\centering
\caption{Predicate definitions for the Kinematic-GNN engine}
\label{tab:predicate-definitions}
\begin{tabular}{lll}
\toprule
\textbf{Category} & \textbf{Predicate} & \textbf{Geometric Condition} \\
\midrule
\multirow{5}{*}{Spatial} 
    & \texttt{is\_near} & Euclidean Distance $< 0.2$m \\
    & \texttt{is\_above} & Relative $\Delta z > 0.1$m \\
    & \texttt{is\_below} & Relative $\Delta z < -0.1$m \\
    & \texttt{is\_left\_of} & Relative $\Delta x < -0.05$m \\
    & \texttt{is\_right\_of} & Relative $\Delta x > 0.05$m \\
\midrule
\multirow{4}{*}{Interaction}
    & \texttt{is\_holding} & Gripper near object AND gripper closed \\
    & \texttt{is\_contacting} & Euclidean Distance $< 0.05$m \\
    & \texttt{is\_approaching} & Velocity vector oriented toward target \\
    & \texttt{is\_retracting} & Velocity vector oriented away from target \\
\bottomrule
\end{tabular}
\end{table}

\section{Training Objective and Heuristic Labeling}
\label{sec:training-objective}

We train with \textbf{Weighted Focal Loss} over all nine predicates to address the extreme class imbalance ($\sim$95\% negative labels). The focal modulating factor $(1 - p_t)^\gamma$ down-weights easy negatives, forcing the model to focus on ambiguous interaction edges. Since the ALOHA dataset lacks annotations, labels are generated heuristically based on kinematic thresholds \cite{zuo_graph-based_2021, zhou_multi-robot_2022, wang_research_2025}. To handle the 55,000-frame dataset, we implement \textbf{RAM Pre-computation}, achieving a \textbf{730$\times$ speedup} in training efficiency.

\section{Agent Reasoning and Benchmarking Methodology}
\label{sec:agent-benchmarking}

To evaluate the core thesis claim of swappable intelligence, we implement a standardized interface for local Large Language Models (LLMs) and a rigorous benchmarking suite.

\subsection{QwenAgent Integration and Prompt Design}

The \texttt{QwenAgent} class facilitates the integration of the Qwen 2.5 (3B) model via the Ollama backend. Unlike previous iterations, the Qwen agent utilizes a simplified, rule-based system prompt designed to improve task completion and reduce instruction-following errors. 

The prompt design enforces a strict \textit{Observation-Thought-Action} loop, where the agent is required to:
\begin{enumerate}
    \item Analyze the world graph retrieved via MCP Resources.
    \item Explicitly state its reasoning (Thought) regarding the physical workspace.
    \item Select a valid MCP tool call from the registered toolset.
\end{enumerate}
This design was specifically optimized to solve the ``infinite loop'' errors observed in earlier Llama 3.2 trials by providing explicit completion conditions when predicates match the user goal.

\subsection{Standardized Agent Benchmark (10-Goal Suite)}

We established a benchmark comprising ten standardized goals to quantitatively assess agentic performance across the MCP-ROS~2 bridge. The methodology evaluates success based on task completion, step efficiency, and total execution time. The goals are designed to test the breadth of the protocol's capabilities:

\begin{enumerate}
    \item \textbf{State Retrieval}: Retrieve the current world graph.
    \item \textbf{Spatial Reasoning}: Report all active spatial predicates in the scene.
    \item \textbf{Entity Analysis}: Count the number of nodes present in the current world graph.
    \item \textbf{Temporal Navigation}: Advance to dataset frame 10 and retrieve active predicates.
    \item \textbf{Belief State Updates}: Set the environment to frame 50 and report the updated graph.
    \item \textbf{Predicate Inventory}: List every active relationship between robot links and objects.
    \item \textbf{Sensitivity Testing}: Retrieve predicates using a customized confidence threshold (e.g., 0.7).
    \item \textbf{Safety Verification}: Simulate a ``move forward'' action and verify the predicted outcome via forward dynamics.
    \item \textbf{Mental Rollout}: Predict the outcome of a 90-degree rotation.
    \item \textbf{Workspace Analysis}: Analyze the scene to identify object-to-object proximity (e.g., ``is cup near machine?'').
\end{enumerate}