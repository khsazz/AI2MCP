% ============================================================================
% Chapter 2: Background
% ============================================================================

\chapter{Background}
\label{ch:background}

This chapter presents the theoretical and technical foundations underpinning this thesis. It introduces the Model Context Protocol (MCP) as a model-facing standard for protocol-driven robotics, situates it with respect to the Robot Operating System 2 (ROS~2), and reviews core concepts in relational reasoning and forward dynamics that support evidence-based agentic planning and swappable intelligence.

\section{The Model Context Protocol (MCP)}

The Model Context Protocol (MCP) is an open standard designed to facilitate seamless communication between Large Language Models (LLMs) and external data sources or tools. Traditionally, AI agents required bespoke connectors for every unique API, leading to the $N \times M$ integration problem characterized in Chapter 1. MCP standardizes this interaction through a JSON-RPC 2.0 based transport layer, typically implemented over Server-Sent Events (SSE) \cite{wang_towards_2025, chhetri_model_2025}.

\subsection{Standardization and Maturity}
While MCP is an emerging standard, its adoption as a "de facto" bridge for AI-driven tool use is accelerating due to its runtime-agnostic design \cite{lee_robot_2025, patil_model_2025}. Unlike previous proprietary interfaces, MCP provides a "universal plug" that allows an AI model to discover hardware capabilities dynamically. While proposals like the \textit{Robot Context Protocol} (RCP) represent the future of multi-tenant robotic control, MCP currently offers the most mature software development kit (SDK) for grounding foundation models in digital and physical tools \cite{lee_robot_2025, chhetri_model_2025}. This thesis commits to MCP as the primary interface to ensure the architecture is compatible with the rapidly evolving ecosystem of "agentic" AI applications.

\section{Robot Operating System 2 (ROS~2)}

The Robot Operating System 2 (ROS~2) is the industry-standard middleware for contemporary robotic systems, providing real-time, distributed communication over a Data Distribution Service (DDS) backbone \cite{gambo_systematic_2025, macenski2022ros2}. ROS~2 standardizes intra-robot messaging through \emph{Topics}, \emph{Services}, and \emph{Actions}, thereby offering a unified abstraction over heterogeneous hardware.

A central challenge in ROS~2 is the high barrier to entry for cloud-based AI models, which typically lack the native environments required to interact with DDS topics. Integration traditionally proceeds via platform-specific bridges that are brittle and difficult to reuse \cite{wirkus_online_2020}. In this work, the proposed MCP--ROS~2 bridge functions as a protocol-level adaptor: it exposes ROS~2 capabilities as standardized Tools and Resources, enabling agentic planners to interact with the robot through a unified interface while preserving the 100+ Hz reflexive control loops required for hardware stability \cite{fu_rosbag_2025, liskova_cps-based_2025}.

\section{Relational Reasoning with Graph Neural Networks}

Relational reasoning is a cornerstone of human-like physical intelligence. Graph Neural Networks (GNNs) generalize deep learning to graph-structured data, providing a natural relational inductive bias over entities and their interactions \cite{huang_planning_2022, zuo_graph-based_2021}.

\subsection{Mapping Kinematics to Semantic Graphs}
Our reasoning methodology involves transforming a 14-Degree-of-Freedom (DoF) joint state vector $\mathbf{s}$ into a directed graph $G = (V, E)$. 
\begin{itemize}
    \item \textbf{Nodes} $V$: Comprise $N=16$ entities (7 joints per bimanual arm plus 2 end-effectors).
    \item \textbf{Edges} $E$: Encode kinematic chain connections and spatial proximity.
    \item \textbf{Predicates}: The model predicts $K=9$ binary spatial and interaction predicates for each edge, such as \texttt{is\_near} and \texttt{is\_left\_of}.
\end{itemize}
By grounding agentic planning in these structured predicates rather than raw coordinates, the system can reason about contact and reachability with a validation accuracy of \textbf{97.03\%} \cite{chen_relgnn_2025}.

\section{Forward Dynamics and the Physicality Filter}

Safe deployment of non-deterministic agents requires mechanisms to predict the physical consequences of actions prior to execution \cite{fisac_general_2017, roth_learned_2025}. This thesis introduces a learned **Forward Dynamics Model** as a novel protocol-level safety primitive. 

Given a sequence of proposed tool invocations, the model performs a "mental rollout" to forecast future world states $G_{t+1}$. By achieving a state-prediction error of $\delta = 0.0017$, this component operates as a \textbf{Physicality Filter}. It allows the MCP bridge to veto agent-generated commands that lead to kinematically impossible or unsafe system configurations, providing an interpretable audit trail of safety decisions within the protocol logs \cite{lee_robot_2025, roth_learned_2025}.

\section{Related Work and Comparative Positioning}

A growing body of work explores the integration of foundation models with robotics. Systems like \textit{SayCan} \cite{saycan2022} and \textit{RT-2} \cite{rt22023} demonstrate that LLMs can produce long-horizon plans. However, these are often tightly coupled to proprietary APIs or lack explicit safety verification loops. Table~\ref{tab:related-work-comparison} contrasts AI2MCP with these existing frameworks.

\begin{table}[htbp]
\centering
\caption{Comparative Analysis of Embodied AI Integration Frameworks}
\label{tab:related-work-comparison}
\resizebox{\textwidth}{!}{%
\begin{tabular}{lccccr}
\toprule
\textbf{System} & \textbf{Protocol} & \textbf{Safety Layer} & \textbf{Swappability} & \textbf{Abstraction} & \textbf{Reasoning} \\
\midrule
SayCan \cite{saycan2022} & Proprietary & Affordance-based & Low & Task-Specific & Language \\
RT-2 \cite{rt22023} & None & Probabilistic & None & Monolithic VLA & Multimodal \\
RCP (Draft) \cite{lee_robot_2025} & RCP & Static Rules & Medium & Middleware & Variable \\
Modular ROS 2 \cite{macenski2022ros2} & DDS/C++ & Hand-coded Rules & Low & Hardware-Centric & Deterministic \\
\midrule
\textbf{AI2MCP (Ours)} & \textbf{MCP (Std.)} & \textbf{Sim-Verification} & \textbf{High (Hot-swap)} & \textbf{Tool-Centric} & \textbf{Kinematic GNN} \\
\bottomrule
\end{tabular}}
\end{table}

\subsection{Novelty and Technical Synthesis}
The novelty of this work lies in the synthesis of a relational reasoning layer and learned forward dynamics as standardized protocol "Tools." While previous works have implemented bespoke GNN reasoning for manipulation \cite{huang_planning_2022, lin_efficient_2021}, this is the first implementation of an MCP-to-ROS~2 bridge that enables **hot-swappable intelligence** while maintaining a throughput of **16.9 operations per second**. This advances the field from platform-specific control to a platform-agnostic, protocol-driven paradigm.